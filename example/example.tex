\documentclass[12pt,fleqn]{article}

\usepackage{amsmath,amssymb,amsthm}
\usepackage{xcolor}
\usepackage[margin=1in]{geometry}
\usepackage[hypertexnames=false,bookmarksnumbered=true,final]{hyperref}
\usepackage[capitalize,sort]{cleveref}

\colorlet{textColor}{black}
\colorlet{bgColor}{white}
\colorlet{textBlue}{blue!50!textColor}
\colorlet{textRed}{red!50!textColor}
\hypersetup{colorlinks,linkcolor=textRed,citecolor=textRed,urlcolor=textBlue}

\newtheorem{theorem}{Theorem}
\newtheorem{definition}{Definition}
\newtheorem{example}{Example}
\newtheorem{lemma}[theorem]{Lemma}

\allowdisplaybreaks
%\algnewcommand{\LineComment}[1]{\State \textcolor{gray}{\texttt{//} \textit{#1}}}
%\renewcommand{\algorithmiccomment}[1]{\hfill\textcolor{gray}{\texttt{//} \textit{#1}}}

\newcommand*{\defeq}{:=}
\DeclareMathOperator{\conv}{conv}

\title{Theorems}
\author{\empty}
\date{\empty}

\begin{document}

\maketitle
\setlength{\parindent}{0pt}
\setlength{\parskip}{0.5em}

\section{Abstract Algebra}
\label{sec:abstract-algebra}

\begin{definition}
\label{defn:monoid}
A monoid $M$ is a set $S$ along with a binary operator $\circ: S \times S \to S$ which satisfies these properties:
\begin{enumerate}
\item \label{item:monoid:assoc}\emph{Associativity}:
    $\forall a, b, c \in S$, $(a \circ b) \circ c = a \circ (b \circ c)$.
\item \label{item:monoid:identity}\emph{Existence of Identity}:
    There exists an element $e \in S$, called an \emph{identity} of $S$, such that
    $\forall a \in S$, $a \circ e = e \circ a = a$.
\end{enumerate}
\end{definition}

\begin{lemma}
\label{thm:monoid-unique-identity}
Every monoid has a unique identity.
\end{lemma}
\begin{proof}
Let $M = (S, \circ)$ be a monoid.
Let $e_1$ and $e_2$ be any two (possibly identical) identities of $M$.
Then $e_1 \circ e_2 = e_1$, because $e_2$ is an identity,
and $e_1 \circ e_2 = e_2$, because $e_1$ is an identity.
Hence, $e_1 = e_2$.
Hence, $M$ has a unique identity
\end{proof}

\section{Convexity}
\label{sec:convexity}

\begin{definition}
\label{defn:convex-hull}
For a finite set $P \defeq \{x_1, \ldots, x_n\}$, the \emph{convex hull} of $P$ is defined as
\[ \conv(P) \defeq \left\{\sum_{i=1}^n \alpha_ix_i \biggm| \alpha_i \in \mathbb{R}_{\ge 0}
    \textrm{ and } \sum_{i=1}^n \alpha_i = 1\right\}. \]
\end{definition}

\begin{theorem}[Carath\'eodory's theorem \cite{caratheodory}]
\label{thm:caratheodory}
Let $x \in \conv(P)$, where $P \subseteq \mathbb{R}^d$ is a finite set.
Then $x \in \conv(Q)$, where $Q \subseteq P$ and $|Q| \le d+1$.
\end{theorem}

\bibliographystyle{plainurl}
\bibliography{bibdb}

\end{document}
